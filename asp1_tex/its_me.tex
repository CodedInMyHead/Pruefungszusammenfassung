\documentclass[asp1.tex]{subfiles}
\begin{document}
\section{ITS Me}
\subsection{Datensicherheit}

Datensicherheit hat generell mit der Sicherheit von Daten zu tun. \\
Ziel der Datensicherheit ist der Schutz von Daten allgemein, nicht nur von personenbezogenen Daten. \\
Das oberste Ziel der Datensicherheit besteht in der Gewährleistung der

\begin{itemize}

    \item Vertraulichkeit
    \item Integrität und
    \item Verfügbarkeit von Daten

\end{itemize}
Datensicherheit handelt sich hier um die praktischen Sicherheitsmaßnahmen oder Ansätze zum Schutz von Daten (z.B. Maßnahmen zur Datensicherung, technischer Schutz vor Datenverlust usw.).


\subsection{Datenschutz}

Unter Datenschutz versteht man den Schutz von personenbezogenen Daten. \\
Ziel des Datenschutzes ist der Schutz des allgemeinen Persönlichkeitsrechts der betroffenen Personen. \\
\begin{itemize}

    \item Schützen der Grundrechte und Grundfreiheiten einer Person.

\end{itemize}

\subsection{BSI (Bundesamt für Sicherheit in der Informationstechnik)}

Im BSI Grundschutzkompendium kann nachgeschlagen werden:
\begin{itemize}
    \item welche möglichen Gefährdungen bestehen
    \item welche Auswirkungen diese haben
    \item wie ein Unternehmen sich dagegen schützen kann
\end{itemize}
Das BSI Grundschutzkompendium ist bei einer Schutzbedarfsanalyse als Nachschlagewerk zu betrachten, jedoch nicht als abgeschlossene Analyse, da diese für jedes Unternehmen unterschiedlich ist.

\subsection{Schutzbedarfskategorien}

\begin{itemize}
    \item normal: Begrenzte und überschaubare Auswirkungen \\
    \item hoch: Beträchtliche Auswirkungen \\
    \item sehr hoch: Kritische, möglicherweise existenzbedrohende Auswirkungen
\end{itemize}

\subsection{Schadensszenarien}

\begin{itemize}
    \item Verstöße gegen Gesetze, Vorschriften oder Verträge
    \item Beeinträchtigungen des informationellen Selbstbestimmungsrechts
    \item Beeinträchtigungen der persönlichen Unversehrtheit
    \item Beeinträchtigungen der Aufgabenerfüllung
    \item Negative Innen- oder Außenwirkung
    \item Finanzielle Auswirkungen
\end{itemize}

\subsection{Schutzziele}

\paragraph{Vertraulichkeit}
Schutz vor Verrat von Informationen oder vertraulichen Daten. \\
Darunter sind z.B. Umsatzzahlen und daraus erhobene Daten oder Produkt- und Forschungsergebnisse zu verstehen. \\
(Interessant für die Konkurrenz)
\begin{itemize}
    \item[=] Vertrauliche Informationen unberechtigt zur Kenntnis genommen oder weitergegeben
\end{itemize}
\paragraph{Integrität}
Das durchgängige Funktionieren von IT-Systemen sowie die Vollständigkeit und Richtigkeit von Daten und Informationen.\\ In Bezug auf die Informationssicherheit bedeutet Integrität das Verhindern von nicht genehmigten Veränderungen an wichtigen Informationen. \\
Integrität von Daten kann man z.B. durch Verschlüsselung erreichen.

\begin{itemize}
    \item[=] Die Korrektheit der Informationen und der Funktionsweise von Systemen ist nicht mehr gegeben
\end{itemize}
\paragraph{Verfügbarkeit  von Daten}
Definiert, zu welchem Grad IT-Systeme, IT Anwendungen, IT Netzwerke und elektronische Informationen einem Benutzer zur Verfügung stehen und ohne Einschränkung verwendet werden können.

\begin{itemize}
    \item[=] Autorisierte Benutzer werden am Zugriff auf Informationen und Systeme behindert
\end{itemize}

\subsection{DSGVO - Datenschutz-Grundverordnung}

Die EU-Datenschutzgrundverordnung (DSGVO) ist unmittelbar geltendes Recht. Sie hat Vorrang vor nationalem Recht. An manchen Stellen lässt sie aber einzelne Bestimmungen offen (Öffnungsklauseln). Diese werden durch das Bundesdatenschutzgesetz (BDSG) konkretisiert.

\subsection{Grundsätze der DSGVO}

\textbf{Rechtmäßigkeit}, Verarbeitung nach Treu und Glauben, Transparenz
Personenbezogenen Daten dürfen nur so verarbeitet werden, wie es bei der Erhebung angegeben wurde. Zudem muss die Verarbeitung in einer für den Betroffenen nachvollziehbaren Weise erfolgen. Es sind keine verdeckte oder geheime Verarbeitung erlaubt. Die betroffene Person sollte wissen, wer der Verantwortliche für die Verarbeitung ist.

\textbf{Zweckbindung}
Personenbezogene Daten dürfen nur für den Zweck verarbeitet werden, für den sie erhoben wurden.
Beispiel:
Ein Kunde widerruft seine Einwilligung zum Erhalt von Newslettern. Ab diesem Zeitpunkt ist der Zweck der Datenverarbeitung nicht mehr gegeben. Somit darf das Unternehmen die Daten der betroffenen Person nicht mehr verarbeiten.

\textbf{Datenminimierung}
Unternehmen dürfen nur so viele Daten erheben und verarbeiten, wie sie tatsächlich benötigt. Die Daten müssen für den Zweck erheblich und relevant sein.
Beispiel:
Für den Abschluss eines Kaufvertrages darf die Religionszugehörigkeit oder der Familienstand nicht zusätzlich erhoben werden. Denn diese spielen für den Kauf keine wesentliche Rolle.

\textbf{Richtigkeit}
Daten müssen inhaltlich und sachlich richtig und aktuell gehalten werden.

\textbf{Speicherbegrenzung}
Speicherbegrenzung bezieht sich auf die Dauer der Speicherung. Daten dürfen nicht für die Ewigkeit gespeichert werden. Ist der Zweck nicht mehr gegeben, müssen die Daten gelöscht werden.

\textbf{Integrität und Vertraulichkeit}
Die Daten müssen vor unrechtmäßiger Verarbeitung durch Unbefugte geschützt werden. Ebenso müssen die Daten vor versehentlicher Beschädigung oder Verlust geschützt werden.

\textbf{Rechenschaftspflicht}
In der DSGVO gilt die Rechenschaftspflicht, d.h. die verantwortliche Stelle ist für die Einhaltung der oben genannten Grundsätze verantwortlich. Auf Verlangen muss sie die Einhaltung gegenüber den Betroffenen und den Behörden nachweisen können.

\subsection{Schutzbedarfsanalyse}

Sicherheitskonzept, das drei Schutzzielen, deren potenziellen Gefährdungen und die Auswirkungen / Maßnahmen dagegen dokumentiert. \\
\\
\textbf{Verlust der Verfügbarkeit von Informationen} \\
\textbf{Gefährdung}: Verschleiß von Festplatten, die Daten speichern.\\
\textbf{Auswirkungen}: Totaler Ausfall \\ (die Daten werden für den laufenden Geschäftsbetrieb dauerhaft benötigt) \\
\textbf{Maßnahmen dagegen}:
\begin{itemize}
    \item Daten auf mehreren Festplatten speichern (RAID)
    \item Festplatten regelmäßig überprüfen und ggf. austauschen
    \item Daten in verschiedenen Rechenzentren speichern und aktuell halten
\end{itemize}
\vspace{1cm}
\textbf{Verlust der Vertraulichkeit von Informationen} \\
\textbf{Gefährdung}: Sicherheitslücke in einem Rechenzentrum \\
\textbf{Auswirkungen}: Fremde können sich Zugang zu den Servern beschaffen und Daten abgreifen. \\
Verschlechtert das Ansehen des Unternehmens, ggf. rechtliche Konsequenzen \\
\textbf{Maßnahmen dagegen}:
\begin{itemize}
    \item Daten nur verschlüsselt speichern
    \item Eigene Rechenzentren verwenden, um die maximale Kontrolle über Sicherheitsaspekte zu haben
    \item Regelmäßig Sicherheitstests (auch extern)
    \item Zugang zu Servern beschränken
    \item Daten verschieden lagern um Schaden zu begrenzen
\end{itemize}
\vspace{1cm}
\textbf{Verlust der Integrität} \\
\textbf{Gefährdung}: Sicherheitslücke in einem Rechenzentrum \\
\textbf{Auswirkungen}: Unbefugter Zugang zu Servern Manipulation von Daten \\
Fehlerhafte Daten und damit ungewolltes Verhalten der Software \\
\textbf{Maßnahmen dagegen}:
\begin{itemize}
    \item Daten nur verschlüsselt speichern
    \item Eigene Rechenzentren verwenden, um die maximale Kontrolle über Sicherheitsaspekte zu haben
    \item Regelmäßig Sicherheitstests (auch extern)
    \item Zugang zu Servern beschränken
    \item Daten verschieden lagern um Schaden zu begrenzen
    \item RAID-Backups
\end{itemize}
\vspace{1cm}

\subsection{Urheberrecht}

Das Urheberrecht schützt den Urheber in seinen geistigen, persönlichen und vermögensrechtlichen Beziehungen zu seinem Werk, dessen Rechtsschutz mit seiner Entstehung beginnt und im Unterschied zu den gewerblichen Schutzrechten keiner Hinterlegung oder Registrierung bedarf. Als dem Urheberrecht zugängliche Werkarten nennt das UrhG Sprachwerke (Reden, Schriftwerke und Computerprogramme), Werke der Musik, pantomimische Werke und Werke der Tanzkunst, Werke der bildenden und angewandten Kunst, Bauwerke, Lichtbildwerke, Filmwerke sowie Darstellungen wissenschaftlicher und technischer Art (Zeichnungen, Pläne, Karten, Skizzen, Tabellen, plastische Darstellungen).

Neben dem Urheberrecht spielt auch das Markenrecht eine Rolle. Für die Benutzung geschützter Werke benötigt man im geschäftlichen Umfeld in der Regel eine Lizenz.

\subsection{SLA - Service Level Agreement}

Ein SLA ist ein Vertrag zwischen einem Dienstleistungsanbieter und seinen Kunden. \\
Dieser dokumentiert welche Dienstleistungen der Anbieter erbringen wird. \\
Zudem definiert er Dienstleistungsstandards, zu deren Einhaltung der Anbieter verpflichtet ist.

Beinhaltet z.B.:
\begin{itemize}

    \item die Verfügbarkeit
    \item[] Beispiel: $99,7\%$ Verfügbarkeit
          \begin{enumerate}
              \item $365 * 24 * (100-99,7)$  =  Maximale Downtime
              \item Reaktionszeit beachten
              \item Frage: Ist die Zeit überschritten?
              \item Frage: Ist die max. Entstördauer überschritten
          \end{enumerate}
    \item Servicebereitschaft von wann bis wann
    \item Reaktionszeit auf Störungen und maximale Entstördauer
    \item Ggf. Hinweise zum Monitoring von Produkten

\end{itemize}

\subsubsection{First Level Support}

Dieser ist der Erste Ansprechpartner für Beratung und Hilfe im IT- und Computerbereich. \\
Wird er kontaktiert, trägt er zunächst die Daten des Kunden, alle eingehenden Anfragen und weitergehende Informationen zusammen. \\
Anschließend listet er sie auf. \\
Die Dokumentation sollte möglichst lückenlos erfolgen. \\
Dies verhindert unangenehme Nachfragen des Kunden und gewährleistet eine reibungslose Weitergabe der Anfrage an den nächsten Support-Level.  \\
Zunächst kümmert sich aber der First-Level-Supporter eigenständig um das Problem. \\
Neben seiner Erfahrung kann er bei Bedarf auch das Wissen externer Datenbanken zu Rate ziehen. \\
In dieser Phase erfolgt daneben auch eine Einstufung der Probleme, die die Kunden haben.

\subsubsection{Second Level Support}

Dieser ist die zweite Ebene im Kundenservice eines Unternehmens. \\
Er versucht technischer Probleme beim Kunden per Ferndiagnose am Telefon oder per Internet-Online Support zeitnah zu lösen. \\
Second Level Supporter müssen über entsprechendes Fachwissen verfügen, das beim First Level Support nicht zwingend erforderlich ist.


\subsubsection{Third Level Support}

Dieser besteht ausschließlich aus Experten und Spezialisten. \\
Diese Stufe wird nur eskaliert, wenn der Second Level keine Lösung gefunden hat.


\subsection{ITIL  (Information Technology Infrastructure Library)}

Der \textbf{ITIL}  ist der \textbf{Best-Practice-Leitfaden} und der \textbf{De-facto-Standard} im Bereich IT-Service-Management. \\
Er besteht nicht aus starren Vorgaben, sondern ist vielmehr eine Sammlung von Leitlinien, die an die Anforderungen einzelner Unternehmen angepasst werden können.
\subsection{Kategorisierung}
\subsubsection{Service Request}

\textbf{Service Requests} sind formale Anfragen eines Anwenders nach etwas, das bereitgestellt werden soll. \\
\begin{itemize}
    \item Anfrage nach Informationen
    \item Beratung
    \item Passwort zurücksetzen
    \item Arbeitsplatz für einen neuen Anwender aufsetzen
\end{itemize}

\subsubsection{Event}
\textbf{Events} sind einfache Benachrichtigungen, die keinen Fehler darstellen müssen \\
\begin{itemize}
    \item Statusänderung
    \item Alarm
    \item Benachrichtigung
\end{itemize}

\subsubsection{Incident}
Ein Incident ist eine Beeinträchtigung oder Unterbrechung eines angebotenen Service. \\
Eine Beeinträchtigung liegt dann vor, wenn der Service nach der Vereinbarung zwischen Servicegeber und Servicenehmer (SLA) quantitativ oder qualitativ nicht wie vereinbart genutzt werden kann.
\end{document}