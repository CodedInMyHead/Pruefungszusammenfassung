\documentclass[asp2.tex]{subfiles}
\begin{document}

\section{VLANs}

\subsection{VLANs erstellen}

Über das Cisco CLI kann mit folgendem Befehl ein VLAN erstell und benannt werden

\begin{lstlisting}
    S1(config)# VLAN 10
    S1(config-vlan)# name Einkauf
\end{lstlisting}

\subsection{VLANs anzeigen}

Folgender command zeigt die bekannten VLANs:

\begin{lstlisting}
    S1(config)# do show vlan brief
\end{lstlisting}

\subsection{Switchmodes}

Eine Verbindung ist bei Cisco immer in einem Modus, hierbei gibt es mehrere, 
die jedoch effektiv bewirken, dass die Verbindung je nach Situation in 
einem von 2 Modi läuft, die im folgenden erläutert werden.
Diese Modi sind relevant für VLANs. 
Daher sollte man, wenn man mit VLANs arbeitet immer expliziz access oder trunk auswählen.

\subsubsection{access}

Access Mode heißt, dass immer 1 VLAN über diese Verbindung gehen kann. Daher muss dieses exlizit angegeben werden.
In diesem Beispiel ist also auf interface fa0/1 nur Vlan 10 erlaubt.

\begin{lstlisting}
    S1(config)# interface FastEthernet0/1
    S1(config-if)# switchport mode access
    S1(config-if)# switchport access vlan 10
\end{lstlisting}

\subsubsection{trunk}

Trunk Mode heißt, dass mehrere VLANs über diese Verbindung gehen können. Diese werden mit Tags unterschieden.
In diesem Beispiel wird interface g0/1 als trunk configuriert. 
ungetaggte Packete werden vlan 99 zugewiesen (native)
Dann wird erst gar kein VLAN zugelassen (reset)
Dann werden VLAN 10,20 und 30 erlaubt

\begin{lstlisting}
    S1(config)# int g0/1
    S1(config-if)# switchport mode trunk
    S1(config-if)# switchport trunk native vlan 99
    S1(config-if)# switchport trunk allow vlan none
    S1(config-if)# switchport trunk allow vlan 10,20,30
\end{lstlisting}

\subsection{Router on a Stick / Lollypop}

"Router on a Stick" ist eine Netzwerkkonfiguration, bei der ein Router dazu verwendet wird, 
den Datenverkehr zwischen verschiedenen VLANs in einem Netzwerk zu ermöglichen.
Hier ist der Router über nur 1 Schnittstelle mit einer Switch verbunden.
Diese Verbindung muss ein Trunk sein.
Der Router mus VLAN Tagging unterstützen und so konfiguriert sein, dass er Pakete weiterleitet.
Um mit den verschiedenen VLANs zu kommunizieren, erstellt der Router logische Sub-Interfaces 
auf seiner physischen Schnittstelle, wobei jedes Sub-Interface einem VLAN entspricht.

In diesem Beispiel wird 1 logisches Interface für das Physische Interface g0/0/0 erstellt.
Diesem wird dann VLAN 10 zugewiesen (Zeile 2)
Zudem bekommt dieses eine eigene Addresse

\begin{lstlisting}
    R1(config)# int g0/0/0.10
    R1(config-subif)# encapsulation dot1q 10
    R1(config-subif)# ip address 192.168.10.1 255.255.255.0
\end{lstlisting}

\subsection{EtherChannel}

Ein EtherChannel ist eine Technologie, die es ermöglicht, mehrere physische Ethernet-Verbindungen
 zwischen zwei Geräten zu einem logischen Bündel zusammenzufassen. 
 Dieses Bündel verhält sich dann wie eine einzige logische Verbindung mit höherer Bandbreite und Redundanz.

 Hier werden g0/1 und g0/2 in "port-channel 1" gelegt. dann wird diese logische Gruppe als trunk 
 gekennzeichnet und normal configuriert

\begin{lstlisting}
    S1(config)# interface range g0/1 - 2
    S1(config-range-if)# channel-group 1 mode on
    S1(config-range-if)# exit
    S1(config)# int port-channel 1
    S1(config-if)# switchport mode trunk
    S1(config-if)# switchport trunk native vlan 99
    S1(config-if)# switchport trunk allowed vlan none
    S1(config-if)# switchport trunk allowed vlan 10,20,30
\end{lstlisting}

\subsection{Multilayer Switch / Layer 3 Switch}

In folgenden wird g1/0/1 mit access mode auf vlan 10 configuriert und eine ip zugewiesen.

\begin{lstlisting}
    S1(config)#int g1/0/1
    S1(config-if)# switchport mode access
    S1(config-if)# switchport access vlan 10
    S1(config-if)# int vlan 10
    S1(config-if)# ip add 192.168.1.1 255.255.255.0
\end{lstlisting}

Dann wird auf der Multilayer Switch das Routing aktiviert

\begin{lstlisting}
    S1(config)# ip routing
\end{lstlisting}

Dann wird eingestellt, dass die Ports g1/1/1-4 nicht mehr als Switch-Ports sondern als Router-Schnittstellen agieren.

\begin{lstlisting}
    S1(config)# int ra g1/1/1-4
    S1(config-range-if)# no switchport
\end{lstlisting}

Diese brauchen dann natürlich noch eine IP und hier eine ospf cost (Niedriger sagt der Switch die Verbindung sei besser)

\begin{lstlisting}
    S1(config)# int g1/1/1
    S1(config-if)# ip add 192.168.6.1 255.255.255.252
    S1(config-if)# ip ospf cost 40
    S1(config)# int g1/1/2
    S1(config-if)# ip add 192.168.7.1 255.255.255.252
    S1(config-if)# ip ospf cost 10
\end{lstlisting}

\end{document}